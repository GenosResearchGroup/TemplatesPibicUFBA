%% Local IspellDict: brasileiro
\documentclass[11pt]{article}
\usepackage{graphicx}
\usepackage{url}
\usepackage[utf8x]{inputenc}
\usepackage[T1]{fontenc}
\usepackage[brazil]{babel}
\usepackage{times}
\usepackage{pibic}

\newcommand{\eng}[1]{\textit{#1}}
\newcommand{\opus}[1]{\textit{#1}}
\newcommand{\ok}{
  \multicolumn{1}{>{\columncolor[gray]{.6}}c}{}
}
\newcommand{\red}[1]{{\color{red}{[#1]}}}

%% http://lattes.cnpq.br/3672057701593524

\begin{document}

\cabecalho{Projeto de Pesquisa do Orientador}

\dadosProjetoOrientador
{Título do projeto}
{Nome do orientador}
{Grupo de pesquisa}
{palavras-chave}
{Dados do edital}
{ano}

\newpage

\singlespace

\Section{Objetivos e Justificativa}

\paragraph{Objetivos}
\label{sec:objetivos}

\paragraph{Justificativa}
\label{sec:justificativa}


\Section{Metodologia}
% Descrição da maneira como serão desenvolvidas as atividades para se
% chegar aos objetivos propostos. Indicar os materiais e métodos que
% serão usados.

\Section{Viabilidade e financiamento}
% Argumentação clara e sucinta, demonstrando a viabilidade do projeto
% e seus financiamentos (se existentes) com fonte e período de
% execução.

\Section{Resultados e impactos esperados}
% Relação dos resultados ou produtos que se espera obter após o
% término da pesquisa.


\Section{Cronograma de execução}
% Relação itemizada das atividades previstas, em ordem seqüencial e
% temporal, de acordo com os objetivos traçados no projeto e dentro do
% período proposto.

Conforme a metodologia, as atividades previstas deste projeto
ocorrerão simultaneamente de modo contínuo. Dessa forma a maioria das
atividades ocorrerão durante todo o seu período de vigência.

\begin{tabular}{l|cccc}
  & \multicolumn{4}{c}{Trimestres}\\
  Atividades& 1 & 2 & 3 & 4 \\
  \hline
  X&\ok&\ok&\ok&\ok\\

\end{tabular}
\addcontentsline{toc}{section}{\refname}

% Relação itemizada das referências que subsidiam a proposta de
% pesquisa, colocando as mais importantes. maximo de 10

\renewcommand{\refname}{
  \hrule\vspace{1em}
  \hspace{1.5em}{\hspace{.5em}Referências
    Bibliográficas (máximo de 10 referências)}
  \vspace{1em}\hrule
}

%\nocite{}

% \bibliographystyle{plain}
% \bibliography{references}

\Section{Detalhamento dos orientandos de graduação}

\vspace{3em}\hspace{-2em}
Salvador, \today{}

\vspace{3em}\hspace{-2em}
Nome do orientador

\end{document}
